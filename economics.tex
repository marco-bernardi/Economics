% BASIC START FOR TEX DOCUMENT with chapter

\documentclass[12pt]{article}
\usepackage{amsmath,amssymb}
\usepackage{color}
\usepackage{enumitem}
\usepackage{hyperref}

% MAKE TITLE AND AUTHOR
\title{Economics}
\author{
    Marco Bernardi
    \and
    Andrea Auletta
    \and
    Aulo
}

\date{\today}
\makeindex
\begin{document}
\maketitle
\tableofcontents
\section{Source of innovation}
Innovation can arise from many different sources. The most common are:
\begin{itemize}
    \item \textbf{Firms}: well suited to innovation, they have greater resources than individuals and management system to control those resources towards a collective purpose.
    Firms face strong incentives to develop differentiating products and services (with respect to no-profit organizations and governments fundend entities);
    \item \textbf{Individuals}: as a loan inventor or users who design solutions to their own problems;
    \item \textbf{Private non-profit organizations};
    \item \textbf{Universities and governments fundend research}: An even more important source arise from the link between universities and governments.
\end{itemize}
\subsection{Creativity}
The Creativity is the ability to generate new and useful ideas or the ability to produce work that is useful and novel.
\begin{itemize}
    \item \textbf{Individual creativity}: is a function of the intelectual abilities, knowledge, personality, motivation, and environment;
    \item \textbf{Organizational creativity}: is a function of creativity of individuals within the organization plus social processes and contextual factors that shape how those individuals 
    interact and behave.\\
    Methods of encouraging organizational creativity include:
    \begin{itemize}
        \item Ideal collection systems: easy to implement, but only a 1st step in unleashing employee creativity;
        \item Creativity training programs;
        \item Colture that encourages creativity;
    \end{itemize}
\end{itemize}
\subsection{Creativity into innovation}
The innovation is the implementation of creative ideas into new devices or processes (creativity + resources and expertise).
\subsubsection{Inventors}
The inventors have mastered the basic tools and operations of the field in which they invent.
Inventors are curios and more interested in problems than in solutions, they question the assumption made in previous work in the field.
Finally, they seek global solutions, not just local ones.
\subsubsection{Innovation by users}
The users create solution for their own needs, and for their own use.
\subsubsection{Research and development by firms}
\begin{itemize}
    \item \textbf{Research}: we've two types of research, the basic research which aims at increasing the understanding of a topic or field without an immediate commercial application, 
    and the applied research which aims at increasing understanding of a topic or field to meet a specific need;
    \item \textbf{Development}: refers to activities that apply knowledge to produce useful devices, materials, or processes;
\end{itemize}
Two approaches to innovation are:
\begin{itemize}
    \item \textbf{Science-push}: innovation proceeds linearly from scientific discovery, invention, manufacturing, and marketing;
    \item \textbf{Demand-pull}: innovation originates with customers needs, from customers suggestions, inventions and manufacturing;
\end{itemize}
\subsubsection{Firm linkages with customers, suppliers, competitors, and complementors}
\begin{itemize}
    \item \textbf{External versus internal sources of innovation}: firms with in-house R\&D are more heaviest users of external collaboration networks;
    And may help firms to build \textbf{absorptive capacity} that enables them to a better use of the information obtained from external sources;
    \item \textbf{Universities and governments fundend research}: Universities encourage research but have smaller resources, also contribute to innovation trough pubblication of the results.
    Governments do research through own laboratories, science parks, incubators, and grants for other public or private research organizations;
    \item \textbf{Private non-profit organizations}: in house R\&D or fund R\&D in other organizations or both.
\end{itemize}
\subsection{Innovation in collaborative networks}
It is important in high technology sectors, where individual firms rarely posses all necessary resources and capabilities to innovate.
With a lot of relationships, can be that there are larger diffusion of information and resources.
There are \textbf{technology clusters} which are regional clusters\footnote{group of individuals having same characteristics} or firms that have a connection to a common technology.
Likelihood of innovation activities being geographically clustered depends on: the nature of the technology, the industry characteristics,  and cultural context of technology.

\textbf{Technological spillovers} occur when the benefits from the research actievities of one firm spill over to other firms, and they are in function of
strength of protection mechanisms, nature of underlying knowledge base and mobility of the labor pool.
\section{Types and patterns of innovation}
We've different dimensions that are often used to classify technology:
\begin{itemize}
    \item \textbf{Product Innovation}: it is the development of new products or the enhancement of existing ones. The success of many companies is often tied to their ability to innovate in the products they offer;
    \item \textbf{Process Innovation}: it is the implementation of a new or significantly improved production or delivery method. It can also involve substantial changes in techniques, equipment and/or software;
\end{itemize}
We can also distinguish them according to the impact they have:
\begin{itemize}
    \item \textbf{Incremental Innovation}: it is the gradual improvement of existing products, services, processes or methods. It is often driven by customer feedback and is aimed at maintaining or improving a company's competitive position;
    \item \textbf{Radical Innovation}: it is the development of new products, services, processes or methods that are significantly different from existing ones. It is often driven by technological advances and can lead to the creation of new markets or the disruption of existing ones;
\end{itemize}
And according to the impact on the internal organization:
\begin{itemize}
    \item \textbf{Competence Enhancing}: it is based on the firm's existing knowledge and capabilities, and aim to enhance and extend them;
    \item \textbf{Competence Destroying}: it is based on the firm's existing knowledge and capabilities, and aim to render them obsolete for replacing them with new ones.
\end{itemize}
We can take the products themselves and also categorize:
\begin{itemize}
    \item \textbf{Component Innovation}: it entails changes to one or more components of a product system without significantly affecting the overall design;
    \item \textbf{Architectural Innovation}: it entails changing the overall design of the system, or the way components interact.
\end{itemize}
There are also entirely new fields of technology, like \textbf{social innovation} which is the development of new ideas, services, or models that better meet social needs and create new social relationships or collaborations.
\subsection{Technological S-curve}
S-curves describes two types of ratio:
\begin{itemize}
    \item \textbf{Rate of a technology's performance improvement}: if the technology is very different from previous technologies, there may be no evaluation routines that enable researchers 
    to assess it's progress or potential. Furthermore, until the technology has established a degree of legitimacy, it may be difficult to attract other researchers to partecipate in its development.
    The technology begins to reach it's inherent limits, the cost increase and the s-curve flattens;
    Technologies do not always reach their limits: they may be rendered obsolete by discontinuous technology\footnote{it fullfills a similar market need, 
    but until the construction of a new knowledge base}.
    \textbf{Disadvantages}: is rare that the true limit is known in advance.
    \item \textbf{Rate of technology's market penetration}: the s-curve is represented by the cumulative number of adopters against time.
    We have three phases:
    \begin{enumerate}
        \item \textbf{Slow growth}: the technology is not well understood, and the potential adopters are not yet convinced of its value;
        \item \textbf{Rapid growth}: the technology becomes better understood, and the potential adopters become convinced of its value;
        \item \textbf{Mature phase}: the technology is well understood, and the market is saturated.
    \end{enumerate}
    \textbf{Disadvantages}: the shape of the curve is not set in the stone.
\end{itemize}
\section{Standards battles, modularity, and platform competition}
\textbf{Dominant design}: is a specific configuration, standard or a set of features that naturally becomes the dominant standard in the market.
It represents a common set of design features that are widely accepted by the market and that are used by a large number of firms.
Once a dominant design is established, the market becomes more predictable and stable, providing a foundation for the development of complementary products and services.
\subsection{Why dominant designs are selected}
Increasing returns to adoption: the more people use a product, the more valuable it becomes.
Two of the primary sources of increasing returns are:
\begin{itemize}
    \item \textbf{Learning effects}: the more a technology is used, the more it is developed and the more effective and efficient it becomes.
    Prior learning and absorptive capacity: a firms investement in prior learning can accelerate it's learning in the future and it's ability to absorb new knowledge (absorptive capacity\footnote{the ability of an organization to recnogize, assimilate, end utilize new knowledge});
    \item \textbf{Network Externalities}: this is when the value of a good to an user increases with the number of other users of the same or similar good.
    The number of users of a technology is often refereed to as the installed base\footnote{the number of users of a particular good}.
    Network externalities also arise when complementary goods\footnote{additional goods and services that enable or enhance the value of another good} are important.
\end{itemize}
\subsection{Multiple dimensions of value}
\subsubsection{Technology standalone value}
It refers to the things it can do or the sources of appeal that are not due to it's installed base or available complements.
\subsubsection{Network externalities value}
Value of the technological innovation to users will be a function not only on it's standalone benefits and costs, but also of the value created by the size
of the installed base and the availability of complements.
\subsection{Modularity and platform competition}
Product may be made increasingly modular, both by expanding the range of compatible components and by uncoupling integrated functions within components.
\textbf{Advantages}:
\begin{itemize}
    \item Often offer more choice over function, design, scale and other features enabling the customer to choose a product suited to their specific needs;
    \item components are reused in different combinations, and can achieve product variety.
\end{itemize}
\section{Timing of entry}
A technology that is adopted earlier than others may reap self-reinforcing advantages:
\begin{itemize}
    \item greater funds to invest in improving the technology;
    \item greater availability of complementary goods;
    \item less customers uncertainty.
\end{itemize}
But if there are a few users and the availability of complementary goods is low, the technology may not be able to reach the customer.
We've 3 types of entrants:
\begin{itemize}
    \item \textbf{First movers (pioneers)}: they are the first to enter a market;
    \item \textbf{Early followers}: early to the market, but not first;
    \item \textbf{Late entrants}: they enter the market after the technology begins to penetrate the mass market.
\end{itemize}
\subsection{First movers advantages}
\begin{itemize}
    \item \textbf{Brand loyalty and technological leadership}: long-lasting reputation as a leader in that technology domain and this can help to sustain the company's image,
    brand loyalty and market share.
    Organization's position as a technological leader also enables it to shape customer expectations about form, features, etc.
    If the aspects are difficult for the competitors to imitate can be there a \textbf{monopoly rents}, otherwise the first movers has the opportunity to build brand loyalty before the entry of other competitors;
    \item \textbf{Preemption of scarce assets}: the enter can capture scarce resources like key location, governament permits, patents, acces to distribution channels and relationships with suppliers;
    \item \textbf{Exploiting buyer switching costs}: if buyers face switching costs, the firms that captures customers early may be able to keep those customers even if technology with superior 
    value is introduced by a competitor;
    \item \textbf{Reaping increasing returns advantages}: in a industry characterized by increasing returns, a technology that is adopted early may raise the market power 
    through self-reinforcing positive feedbacks, mechanisms, culminating in its entrenchment as the dominant design.
\end{itemize}
\subsection{First movers disadvantages}
\begin{itemize}
    \item \textbf{Research and development expenses}: the first movers tipycally bears the brunt of R\&D expenses (technological paths that didn't yield a commercially viable product and complementary
    parts that are not in the market yet).
    Later entrants can also observe the market response to particular features and decide how to focus their R\&D efforts;
    \item \textbf{Undeveloped suppliers and distribution channels}: at the beginning, often, no appropriate suppliers and distribution channels are available;
    \item \textbf{Immature enabling technologies and complements}: firms ofthen rely on other producers of enabling technologies, important complements may not yet be fully developed;
    \item \textbf{Uncertainty of customers requirements}: which features and at what price the customers will want the product.
    First movers can have an opportonity to shape customer preferences.
\end{itemize}
\subsection{Strategies to improve timings options}
A firm which want to early entry in the market needs a fast cycle development process.
The development time can be reduced by using strategic alliances, crossfunctional teams, and parallel development processes.
\end{document}
