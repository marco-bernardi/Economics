% BASIC START FOR TEX DOCUMENT with chapter

\documentclass[12pt]{article}
\usepackage{amsmath,amssymb}
\usepackage{color}
\usepackage{enumitem}
\usepackage{hyperref}

% MAKE TITLE AND AUTHOR
\title{Economics}
\author{
    Marco Bernardi
    \and
    Andrea Auletta
    \and
    Aulo
}

\date{\today}
\makeindex
\begin{document}
\maketitle
\tableofcontents
\section{Source of innovation}
Innovation can arise from many different sources. The most common are:
\begin{itemize}
    \item \textbf{Firms}: well suited to innovation, they have greater resources than individuals and management system to control those resources towards a collective purpose.
    Firms face strong incentives to develop differentiating products and services (with respect to no-profit organizations and governments fundend entities);
    \item \textbf{Individuals}: as a loan inventor or users who design solutions to their own problems;
    \item \textbf{Private non-profit organizations};
    \item \textbf{Universities and governments fundend research}: An even more important source arise from the link between universities and governments.
\end{itemize}
\subsection{Creativity}
The Creativity is the ability to generate new and useful ideas or the ability to produce work that is useful and novel.
\begin{itemize}
    \item \textbf{Individual creativity}: is a function of the intelectual abilities, knowledge, personality, motivation, and environment;
    \item \textbf{Organizational creativity}: is a function of creativity of individuals within the organization plus social processes and contextual factors that shape how those individuals 
    interact and behave.\\
    Methods of encouraging organizational creativity include:
    \begin{itemize}
        \item Ideal collection systems: easy to implement, but only a 1st step in unleashing employee creativity;
        \item Creativity training programs;
        \item Colture that encourages creativity;
    \end{itemize}
\end{itemize}
\subsection{Creativity into innovation}
The innovation is the implementation of creative ideas into new devices or processes (creativity + resources and expertise).
\subsubsection{Inventors}
The inventors have mastered the basic tools and operations of the field in which they invent.
Inventors are curios and more interested in problems than in solutions, they question the assumption made in previous work in the field.
Finally, they seek global solutions, not just local ones.
\subsubsection{Innovation by users}
The users create solution for their own needs, and for their own use.
\subsubsection{Research and development by firms}
\begin{itemize}
    \item \textbf{Research}: we've two types of research, the basic research which aims at increasing the understanding of a topic or field without an immediate commercial application, 
    and the applied research which aims at increasing understanding of a topic or field to meet a specific need;
    \item \textbf{Development}: refers to activities that apply knowledge to produce useful devices, materials, or processes;
\end{itemize}
Two approaches to innovation are:
\begin{itemize}
    \item \textbf{Science-push}: innovation proceeds linearly from scientific discovery, invention, manufacturing, and marketing;
    \item \textbf{Demand-pull}: innovation originates with customers needs, from customers suggestions, inventions and manufacturing;
\end{itemize}
\subsubsection{Firm linkages with customers, suppliers, competitors, and complementors}
\begin{itemize}
    \item \textbf{External versus internal sources of innovation}: firms with in-house R\&D are more heaviest users of external collaboration networks;
    And may help firms to build \textbf{absorptive capacity} that enables them to a better use of the information obtained from external sources;
    \item \textbf{Universities and governments fundend research}: Universities encourage research but have smaller resources, also contribute to innovation trough pubblication of the results.
    Governments do research through own laboratories, science parks, incubators, and grants for other public or private research organizations;
    \item \textbf{Private non-profit organizations}: in house R\&D or fund R\&D in other organizations or both.
\end{itemize}
\subsection{Innovation in collaborative networks}
It is important in high technology sectors, where individual firms rarely posses all necessary resources and capabilities to innovate.
With a lot of relationships, can be that there are larger diffusion of information and resources.
There are \textbf{technology clusters} which are regional clusters\footnote{group of individuals having same characteristics} or firms that have a connection to a common technology.
Likelihood of innovation activities being geographically clustered depends on: the nature of the technology, the industry characteristics,  and cultural context of technology.

\textbf{Technological spillovers} occur when the benefits from the research actievities of one firm spill over to other firms, and they are in function of
strength of protection mechanisms, nature of underlying knowledge base and mobility of the labor pool.
\section{Types and patterns of innovation}
We've different dimensions that are often used to classify technology:
\begin{itemize}
    \item \textbf{Product Innovation}: it is the development of new products or the enhancement of existing ones. The success of many companies is often tied to their ability to innovate in the products they offer;
    \item \textbf{Process Innovation}: it is the implementation of a new or significantly improved production or delivery method. It can also involve substantial changes in techniques, equipment and/or software;
\end{itemize}
We can also distinguish them according to the impact they have:
\begin{itemize}
    \item \textbf{Incremental Innovation}: it is the gradual improvement of existing products, services, processes or methods. It is often driven by customer feedback and is aimed at maintaining or improving a company's competitive position;
    \item \textbf{Radical Innovation}: it is the development of new products, services, processes or methods that are significantly different from existing ones. It is often driven by technological advances and can lead to the creation of new markets or the disruption of existing ones;
\end{itemize}
And according to the impact on the internal organization:
\begin{itemize}
    \item \textbf{Competence Enhancing}: it is based on the firm's existing knowledge and capabilities, and aim to enhance and extend them;
    \item \textbf{Competence Destroying}: it is based on the firm's existing knowledge and capabilities, and aim to render them obsolete for replacing them with new ones.
\end{itemize}
We can take the products themselves and also categorize:
\begin{itemize}
    \item \textbf{Component Innovation}: it entails changes to one or more components of a product system without significantly affecting the overall design;
    \item \textbf{Architectural Innovation}: it entails changing the overall design of the system, or the way components interact.
\end{itemize}
There are also entirely new fields of technology, like \textbf{social innovation} which is the development of new ideas, services, or models that better meet social needs and create new social relationships or collaborations.
\subsection{Technological S-curve}
S-curves describes two types of ratio:
\begin{itemize}
    \item \textbf{Rate of a technology's performance improvement}: if the technology is very different from previous technologies, there may be no evaluation routines that enable researchers 
    to assess it's progress or potential. Furthermore, until the technology has established a degree of legitimacy, it may be difficult to attract other researchers to partecipate in its development.
    The technology begins to reach it's inherent limits, the cost increase and the s-curve flattens;
    Technologies do not always reach their limits: they may be rendered obsolete by discontinuous technology\footnote{it fullfills a similar market need, 
    but until the construction of a new knowledge base}.
    \textbf{Disadvantages}: is rare that the true limit is known in advance.
    \item \textbf{Rate of technology's market penetration}: the s-curve is represented by the cumulative number of adopters against time.
    We have three phases:
    \begin{enumerate}
        \item \textbf{Slow growth}: the technology is not well understood, and the potential adopters are not yet convinced of its value;
        \item \textbf{Rapid growth}: the technology becomes better understood, and the potential adopters become convinced of its value;
        \item \textbf{Mature phase}: the technology is well understood, and the market is saturated.
    \end{enumerate}
    \textbf{Disadvantages}: the shape of the curve is not set in the stone.
\end{itemize}
\section{Standards battles, modularity, and platform competition}
\textbf{Dominant design}: is a specific configuration, standard or a set of features that naturally becomes the dominant standard in the market.
It represents a common set of design features that are widely accepted by the market and that are used by a large number of firms.
Once a dominant design is established, the market becomes more predictable and stable, providing a foundation for the development of complementary products and services.
\subsection{Why dominant designs are selected}
Increasing returns to adoption: the more people use a product, the more valuable it becomes.
Two of the primary sources of increasing returns are:
\begin{itemize}
    \item \textbf{Learning effects}: the more a technology is used, the more it is developed and the more effective and efficient it becomes.
    Prior learning and absorptive capacity: a firms investment in prior learning can accelerate it's learning in the future and it's ability to absorb new knowledge (absorptive capacity\footnote{the ability of an organization to recnogize, assimilate, end utilize new knowledge});
    \item \textbf{Network Externalities}: this is when the value of a good to an user increases with the number of other users of the same or similar good.
    The number of users of a technology is often refereed to as the installed base\footnote{the number of users of a particular good}.
    Network externalities also arise when complementary goods\footnote{additional goods and services that enable or enhance the value of another good} are important.
\end{itemize}
\subsection{Multiple dimensions of value}
\subsubsection{Technology standalone value}
It refers to the things it can do or the sources of appeal that are not due to it's installed base or available complements.
\subsubsection{Network externalities value}
Value of the technological innovation to users will be a function not only on it's standalone benefits and costs, but also of the value created by the size
of the installed base and the availability of complements.
\subsection{Modularity and platform competition}
Product may be made increasingly modular, both by expanding the range of compatible components and by uncoupling integrated functions within components.
\textbf{Advantages}:
\begin{itemize}
    \item Often offer more choice over function, design, scale and other features enabling the customer to choose a product suited to their specific needs;
    \item components are reused in different combinations, and can achieve product variety.
\end{itemize}
\section{Timing of entry}
A technology that is adopted earlier than others may reap self-reinforcing advantages:
\begin{itemize}
    \item greater funds to invest in improving the technology;
    \item greater availability of complementary goods;
    \item less customers uncertainty.
\end{itemize}
But if there are a few users and the availability of complementary goods is low, the technology may not be able to reach the customer.
We've 3 types of entrants:
\begin{itemize}
    \item \textbf{First movers (pioneers)}: they are the first to enter a market;
    \item \textbf{Early followers}: early to the market, but not first;
    \item \textbf{Late entrants}: they enter the market after the technology begins to penetrate the mass market.
\end{itemize}
\subsection{First movers advantages}
\begin{itemize}
    \item \textbf{Brand loyalty and technological leadership}: long-lasting reputation as a leader in that technology domain and this can help to sustain the company's image,
    brand loyalty and market share.
    Organization's position as a technological leader also enables it to shape customer expectations about form, features, etc.
    If the aspects are difficult for the competitors to imitate can be there a \textbf{monopoly rents}, otherwise the first movers has the opportunity to build brand loyalty before the entry of other competitors;
    \item \textbf{Preemption of scarce assets}: the enter can capture scarce resources like key location, governament permits, patents, acces to distribution channels and relationships with suppliers;
    \item \textbf{Exploiting buyer switching costs}: if buyers face switching costs, the firms that captures customers early may be able to keep those customers even if technology with superior 
    value is introduced by a competitor;
    \item \textbf{Reaping increasing returns advantages}: in a industry characterized by increasing returns, a technology that is adopted early may raise the market power 
    through self-reinforcing positive feedbacks, mechanisms, culminating in its entrenchment as the dominant design.
\end{itemize}
\subsection{First movers disadvantages}
\begin{itemize}
    \item \textbf{Research and development expenses}: the first movers tipycally bears the brunt of R\&D expenses (technological paths that didn't yield a commercially viable product and complementary
    parts that are not in the market yet).
    Later entrants can also observe the market response to particular features and decide how to focus their R\&D efforts;
    \item \textbf{Undeveloped suppliers and distribution channels}: at the beginning, often, no appropriate suppliers and distribution channels are available;
    \item \textbf{Immature enabling technologies and complements}: firms ofthen rely on other producers of enabling technologies, important complements may not yet be fully developed;
    \item \textbf{Uncertainty of customers requirements}: which features and at what price the customers will want the product.
    First movers can have an opportonity to shape customer preferences.
\end{itemize}
\subsection{Strategies to improve timings options}
A firm which want to early entry in the market needs a fast cycle development process.
The development time can be reduced by using strategic alliances, crossfunctional teams, and parallel development processes.
\section{Defining the organization's strategic direction}
The first step of formulating technological innovation is to asses the firm's current position and define its strategic direction for the future (articulating on ambitious strategic intent).
A coherent direction leverages and enhances firm's existing competitive position, and provides for the future the development of the firm.
There are several standard tools of strategic analysis for analyzing external and internal environment of the firm.
\subsection{Internal Analysis}
\subsubsection{Porter's five-force model}
It define the actracrtiveness of an industry and firm's opportunity and threats, it identifies ways in which the external forces differentially affect the firm vis-a-vis, and its competitor.
So the objective is to identify threats and opportunities for the firm:
\begin{itemize}
    \item \textbf{The degree of existing rivalry}: depends on the number and relative size of the competitors, the degree to which competitors are differentiated from each other, the demand conditions and the high exit barriers;
    \item \textbf{Threat of potential entrants}: it's influenced by the degree ti which the industry is likely attract to new entrants and the height of entry barriers;
    \item \textbf{Bargaining power of the suppliers}: depends on the degree to which the firms relies on one or a few suppliers which influences the ability to negotiate good terms, the amount of firm which purchaes from suppliers, switching cost and the fact that the firm can or not backword vertically integrate (produce its own suppliers);
    \item \textbf{Bargaining poer of buyers}: depends on the degree to which the firm is reliant on customers, the diversity between products, the switching cost and if can threaten to backword vertically integrate; 
    \item \textbf{Threat of substitutes}: substitutes are products or services that are not considered competitors, but fulfill a strategically equivalent role for the customer, this depends on how the industry is defined;
    \item \textbf{Role of the complements} which enhance usefulness or desirability of a good: depends on how are important are in the industry, whether are differentiate from the products of various rivals and who captures the value offerd by complements. 
\end{itemize}
\subsubsection{Stakeholder Analysis}
Stakeholder models are often used for both strategic and normative purposes: they emphasizes the stakeholder management issues that are likely to impact the firm's finantial performance and help the firm deal with their ethical or moral implications.
It consists in identyifing all the parties that will be affected by the behaviour of the firm: for each party, the firm identifies what that stakeholder's interests are and what resources contribute to the organization.

\subsection{Internal Analysis}
Often begins with identyifing firm's strength and weaknesses examining  each of the activities (primary and support activities) of the value chain.
Identifies which strengths have potential to be a source of sustainable competitive advantage and the resources must be rare, valuable, durable and inimitable.
Some resources are not imitable:
\begin{itemize}
    \item Tacit: cannot be codified in written form;
    \item Path dependent: depend on a particular historical sequence of events;
    \item Socially complex: arise through the complex interaction of multiple people;
    \item Casually ambiguous: it is unclear how the resource originates.
\end{itemize}
\section{Choosing innovation projects}
\subsection{Quantitative methods for choosing projects}
Usually entail converting projects into some estimate of future cash returns and enable managers to use rigorous mathematical and statistical comparisors of projects. \\
\textbf{Advantages and Disadvantages}:
\begin{itemize}
    \item Can provide concrete finantial estimates that facilitate strategic planning and tradeoff decisions;
    \item Can explicitly consider the timing of investment and cash flows and the time value of money and risk (difficult to anticipate returns of the technology);
    \item Can make the returns of the project seem ambiguous;
    \item Discriminate heavilty long-term projects or risky: may fail to capture the importance of the investment decision.
\end{itemize}
\subsubsection{Discounted cash flow methods}
These are methods for assessing whether the anticipated future benefits are large enough to justify expenditure, given the risk.
Take into: payback period, risks, tiem value of money.
\begin{itemize}
    \item \textbf{Net present value (NPV)}: given a level of expenditure, level of cash inflows, discount rate decide what is the worth project.
    Here managers first estimate the cost of the project and the cash flows the project will yield: NVP = Present value of cash inflow - Present value of cash outflows, if this value is >0 will generate wealth.
    \item \textbf{Internal rate of return (IRR)}: given a level of expenditure, level of cash inflow return what is the rate of return that the project yield.
\end{itemize}
\subsubsection{Real options}
\begin{itemize}
    \item \textbf{Based-stock options} is a finantial model.
    A call option on a stock enable an investor to purchase the right to buy the stock at a specified price in the future.
    If in the future the stock is worth more than the exercise price, typically the investor exercise the option by buying the stock otherwise it will not.
    If a the time of the option is exercised, the stock is worth more then the exercise price but not more than the exercise price + the price paid for the original option, typically the investor will exercise the option but loses money (less if allowed the option to expire). 
\end{itemize}
An investor who makes an initial investment in basic R\&D or in breakthough technologies purchases a true call option to later implement that technology should it prove valuable.

\textbf{Advantages and disadvantages}:
\begin{itemize}
    \item Options are valuable where there is uncertainty, and becouse the technology trajectories are uncertain, an option approach may be useful;
    \item Can lead to better investment decision;
    \item Dynamics of technology investments may not conform to the same assumptiona as finantial market.
\end{itemize}
\subsection{Qualitative methods for choosing projects}
\subsubsection{Screening questions}
Screening questions are questions organized into categories for discussing about potential costs and benefits of a project, after creating the list will be a debate or a scoring mechanisms.
This methods do not always provide concrete answers but enable a firm to consider a wider range of issues that may be important int the firm's development decisions.
\subsubsection{R\&D Portfolio}
This is a map according to degree of change and timing cash flows, managers can use this map to compare their desired balance of projects with their actual balance.
There are four types of development projects:
\begin{itemize}
    \item \textbf{Advanced R\&D}: necessary to develop cutting-edge strategic technology;
    \item \textbf{Breakthough}: involve developmentof products that incorporate revolutionary new product and process technologies;
    \item \textbf{Platform}: offer fundamental improvements in the cost, quality, and performance of a tecnhology over previous generations;
    \item \textbf{Derivative projects}: involve incremental changes in products and/or processes. 
\end{itemize}
Companies that use this method categorize all their projects by resources they require and by how they contribute to the company's product line.
This encourages the company to consider both short-term cash flow needs and long-term strategic momentum in budgeting and planning.
\subsection{Combining quantitative and qualitative information}
\subsubsection{Conjoint analysis}
Here we have a family of techniques used to estimate the specific value that individuals attribute to some attributes of a choice.
It enables to derive strategically the weight to assess and a subjective assessment of complex decision to be decomposed into quantitative scores of relative importance of different criteria.
\subsubsection{Data development analysis (DEA)}
It's a method of assessing a potential project using multiple criteria that may have different kinds of measurment units.
\section{Collaboration strategy}
Firms with collaboration can achieve more, at faster rate, and with less cost or risk than they can achieve alone.
\subsection{Reasons for going solo}
\begin{itemize}
    \item \textbf{Availability of capabilities}: the decision to collaborate depends on the degree to which it posses all of the necessary capabilities;
    \item \textbf{Protecting proprietary technologies}: working closely might expose the company's existing proprietary technology;
    \item \textbf{Controlling technology development and use}: firms desire to have complete control over theri development process and the use of any resulting technology;
    \item \textbf{building and renewing capabilities}: challenge in developing new skills, resources and market knowledge.
\end{itemize}
\subsection{Advantages of collaborating}
\begin{itemize}
    \item \textbf{Acquiring capabilities and resources quickly}; 
    \item \textbf{Increasing flexibility}: collaboration reduces the company's capital commitment and increases its flexibility; 
    \item \textbf{Learning from partners}: can facilitate both the transfer of knowledge between firms and the creation of new knowledge that individual firms could have created alone; 
    \item \textbf{Resource and risk pooling}: share costs and risk;
    \item \textbf{Building a coalition around a shared standard}. 
\end{itemize}
\subsection{Types of collaborative arragements}
The mode of collaboration influences success according to the intended goal and vision to achieve and on this we can find different types of collaborative arrangements
\begin{itemize}
    \item \textbf{Strategic alliances}: whether formal or informal, emphasize combining complementary capabilities or transferring specific skills, fostering synergistic efforts, joint innovation and new market opportunities, both as individual alliances and network of alliances. The choice between one of them depends on the firm’s objectives and has to be carefully managed;
    \item \textbf{joint ventures}:  which require equity investment and the creation of a separate legal entity, allowing partners to share resources, risks and rewards. This necessitates a good degree of governance to navigate conflicts and possibly further enticing new modes of collaboration;
    \item \textbf{Licensing}: ensures that a party grants others the right to use an intellectual personal property and gives access to external technology, while providing clear contractual agreement and effective communication between licensor and licensee, allowing to gain access to valuable technology or content. This achieves success when technologies have a good level of control and standardization;
    \item \textbf{Outsourcing}: means procuring services or good externally from an in-house production, offering efficiency and cost savings while also needing a careful management to prevent overreliance, compromising the firm’s core competencies. The success in this relieson selecting reliable partners and maintaining a strategic focus on core activities;
    \item \textbf{Collevtive research organizations}: collective research organizations facilitate collaboration across multiple firms, hinging success only when open communication, shared goals and effective resource management and pooling is employed.
\end{itemize}

\subsection{Choosing and monitoring partners}
Gaining access to another firm's skills or resources is through collaboration is not without risk.
\subsubsection{Partner selection}
A number of factors can influence how well suited partner are to each others, we consider two dimensions:
\begin{itemize}
    \item \textbf{Resource fit}: degree to which potential partners have resources that can be effectively integrated into a strategy that creates value (often complementary and supplementary resources);
    \item \textbf{Strategic fit}: degree to which partners have computable objectives and styles.
\end{itemize}
\subsubsection{Partner monitoring and governance}
\begin{itemize}
    \item \textbf{Allaince contracts}: legally binding contractual agreements to ensure that partners are fully aware of their rights and obligations in the collaboration and have legal remedies should a partner breach the agreement;
    \item \textbf{Equity ownership}: each partner contributes capital and owns a share of the alliance's capital;
    \item \textbf{Renational governance}: self-enforcing governance is based on the goodwill, trust and reputation.
    This reduce monitoring costs, facilitate more extensive cooperations, sharing and learning.
\end{itemize}
\section{Protecting innovation}
\subsection{Appropriability}
The appropriability is the degree to which a firm can capture the rents from its innovation. In general is determined by how easily or quickly competitors can imitate the innovations.
The knowledge underlying the technology may be rare or difficult to replicate(skills or talent) if is tacit or socially complex.
\subsection{Patents, Trademarks and Copyrights}
\subsubsection{Patents}
Patents protect innovations, they provide exclusive rights for a specified duration, 
preventing others from selling, making or using the patented invention and usually requires 
creating something novel, useful and effective, while at the same time being not obvious. 
To protect such, different kind of patents exist to be more granular according to the specific 
context, but also patent laws, designed to protect and harmonize priorities or reached countries.

\subsubsection{Trademarks}
Trademarks protects words or symbols. Trademarks distinguish different sources of goods from 
one party from goods of another and establish legitimate use of marks and registration of such; 
there exist systems able to simplify registration, for example the Madrid agreement or the 
Madrid protocol, providing treaties and systems to secure and manage protection efficiently 
internationally.

\subsubsection{Copyrights}
Copyright protects an original, asrtistic or literary work. It is usually granted to works of 
authorship, prohibiting others from reproducing something in any way. 
This is usually granted for something like 70 years after 1978 inventions 
and there exist levels of protection like Berne Convention which recognizes automatic protection 
upon creation and differentiated treatment for exceptions and limitations.

\subsection{Trade secrets}
Trade secrets are confidential information belonging to the company and are generally unknown
to others, allowing classes of assets and assets to be protected and shared in disclosure.
This is particularly useful for complex technologies, such as reverse engineering ones,
but also others that require particular production/commercial processes.
They can have an infinite lifespan, as something is simply held internally and requires no formality
contract or registration.
They can also be useful for protecting non-technical information such as business practices and be
exploited to gain a competitive advantage, so they can be useful for protecting unique assets
practices that meet particular criteria.

\subsection{Wholly proprietary systems versus Wholly open systems}
\begin{itemize}
    \item \textbf{Wholly proprietary systems} are proprietary-owned and protected by patents, copyright, secrecy and other mechanisms.
    They are produced and aumented only by their developers. They may be difficult to adopt easily by customers due to higher costs and the inability to mix components;
    \item \textbf{Wholly open systems}: here the techonologies are not protected by patent or secrecy. Freely accessed, augmented and distribuited by anyone;
    \item \textbf{Many technologies are partially open}: Here are used different degrees of control mechanisms. It permits to facilitate the development of the complementary goods provider (license them);
\end{itemize}

\subsection{Advantages of protection and diffusion}
\begin{itemize}
    \item \textbf{Protecion} offer a freater rent of appropriability. Architectural control: ability to determine the structure of the technology, and its compatibility with other goods and decide the rate at which the technology is upgraded or refined.
    If the technology is chosen as dominant design influence over the entire industry;
    \item \textbf{Diffusion} May accrue more rapid adoptions: can stimulate the growth of the installed base and availability of complementary goods.
    Open technologies can also benefit from the collective development efforts of those external to the sponsoring company, the risks here are:
    \begin{itemize}
        \item lack of coordination of internal development;
        \item if improvements get incorporated into the technology and disseminated to other users can be very problematic;
    \end{itemize}
    Given the range of advantages a firms must carefully consider the following factors in deciding whether, and to what degree it should protect its innovation:
    \begin{itemize}
        \item Production capabilities, marketing capabilities and capital;
        \item Industry opposition against sole-source technology;
        \item Resources for internal development;
        \item Control over fragmentation;
        \item Incentives for architectural control.
    \end{itemize}
\end{itemize}



\end{document}





